% Options for packages loaded elsewhere
\PassOptionsToPackage{unicode}{hyperref}
\PassOptionsToPackage{hyphens}{url}
%
\documentclass[
  openany]{book}
\usepackage{lmodern}
\usepackage{amssymb,amsmath}
\usepackage{ifxetex,ifluatex}
\ifnum 0\ifxetex 1\fi\ifluatex 1\fi=0 % if pdftex
  \usepackage[T1]{fontenc}
  \usepackage[utf8]{inputenc}
  \usepackage{textcomp} % provide euro and other symbols
\else % if luatex or xetex
  \usepackage{unicode-math}
  \defaultfontfeatures{Scale=MatchLowercase}
  \defaultfontfeatures[\rmfamily]{Ligatures=TeX,Scale=1}
\fi
% Use upquote if available, for straight quotes in verbatim environments
\IfFileExists{upquote.sty}{\usepackage{upquote}}{}
\IfFileExists{microtype.sty}{% use microtype if available
  \usepackage[]{microtype}
  \UseMicrotypeSet[protrusion]{basicmath} % disable protrusion for tt fonts
}{}
\makeatletter
\@ifundefined{KOMAClassName}{% if non-KOMA class
  \IfFileExists{parskip.sty}{%
    \usepackage{parskip}
  }{% else
    \setlength{\parindent}{0pt}
    \setlength{\parskip}{6pt plus 2pt minus 1pt}}
}{% if KOMA class
  \KOMAoptions{parskip=half}}
\makeatother
\usepackage{xcolor}
\IfFileExists{xurl.sty}{\usepackage{xurl}}{} % add URL line breaks if available
\IfFileExists{bookmark.sty}{\usepackage{bookmark}}{\usepackage{hyperref}}
\hypersetup{
  pdftitle={Get crafty},
  pdfauthor={Heidy Berthoud},
  hidelinks,
  pdfcreator={LaTeX via pandoc}}
\urlstyle{same} % disable monospaced font for URLs
\usepackage{longtable,booktabs}
% Correct order of tables after \paragraph or \subparagraph
\usepackage{etoolbox}
\makeatletter
\patchcmd\longtable{\par}{\if@noskipsec\mbox{}\fi\par}{}{}
\makeatother
% Allow footnotes in longtable head/foot
\IfFileExists{footnotehyper.sty}{\usepackage{footnotehyper}}{\usepackage{footnote}}
\makesavenoteenv{longtable}
\usepackage{graphicx,grffile}
\makeatletter
\def\maxwidth{\ifdim\Gin@nat@width>\linewidth\linewidth\else\Gin@nat@width\fi}
\def\maxheight{\ifdim\Gin@nat@height>\textheight\textheight\else\Gin@nat@height\fi}
\makeatother
% Scale images if necessary, so that they will not overflow the page
% margins by default, and it is still possible to overwrite the defaults
% using explicit options in \includegraphics[width, height, ...]{}
\setkeys{Gin}{width=\maxwidth,height=\maxheight,keepaspectratio}
% Set default figure placement to htbp
\makeatletter
\def\fps@figure{htbp}
\makeatother
\setlength{\emergencystretch}{3em} % prevent overfull lines
\providecommand{\tightlist}{%
  \setlength{\itemsep}{0pt}\setlength{\parskip}{0pt}}
\setcounter{secnumdepth}{5}

\title{Get crafty}
\author{Heidy Berthoud}
\date{}

\begin{document}
\maketitle

{
\setcounter{tocdepth}{1}
\tableofcontents
}
\hypertarget{create-your-open-textbook}{%
\chapter{Create your open textbook}\label{create-your-open-textbook}}

Hello!!

\hypertarget{why-crafts}{%
\chapter{Why Crafts}\label{why-crafts}}

Crafting can be a relaxing and enriching activity. Crafting is a wonderful creative outlet to channel emotions and reduce stress.
This book will introduce several different kinds of crafts:
- fiber arts
- knitting
- crochet
- needlework
- embroidery
- cross stitch

\hypertarget{knitting}{%
\chapter{Knitting}\label{knitting}}

Here is the chapter where I teach you how to knit! It mainly involves visiting lots of yarn stores and filling your home with squashy yarns of varying types and lots of different kinds of needles.
Did you know there are lots of knitting needles? There are:
- \textbf{straight needles} to knit flat objects like scarves
- \textbf{circular needles} to knit in the round for things like sweaters and hats
You can also use circular needles to knit flat and multiple straight needles to knit circular because knitters are creative problem solvers!
There are two basic stitches:
1. knit
2. purl
There are also fancy ways you can combine these stitches, and add in flair like yarn-overs, cables, and drops.
We're going to talk about \emph{intarsia} and \emph{Fair Isle} which are two different forms of color work.
I learned to knit by living in Russia and having my Russian landlady critique my methods with unforgiving bluntness. This is not an option for everyone, so use this book instead!

\end{document}
